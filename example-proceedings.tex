\documentclass[%
  english,%
  todotools=true,%
  trtype=proceedings%
]{hpitr}
\usepackage[backend=biber]{biblatex}
\usepackage[math]{blindtext}
\usepackage{threeparttable}
\usepackage{listings}
\usepackage{algpseudocode}
\DeclareNewTOC[name=Algorithm,
  type=algorithm,
  atbegin=\KOMAoptions{captions=above},
  float]{alg}

\rowcolors*{2}{lightgray!25}{}

\begin{filecontents}{example.bib}
@article{953350,
  Address = {New York, NY, USA},
  Author = {Nassi, I. and Shneiderman, B.},
  Doi = {10.1145/953349.953350},
  Issn = {0362-1340},
  Journal = {SIGPLAN Not.},
  Number = {8},
  Pages = {12--26},
  Publisher = {ACM},
  Title = {Flowchart techniques for structured programming},
  Volume = {8},
  Year = {1973}}
\end{filecontents}
\addbibresource{example.bib}

\begin{document}
\title{Proceedings of the First Workshop on Meaning}
\author{Emilia Editor\and Peter Publisher}
\event{Meaning'99}
\maketitle
\frontmatter
\chapter*{Preface}
\blindtext[2]
\tableofcontents
\lstlistoflistings
\mainmatter

\title{The importance of why and how to do work}
\subtitle{An imaginary paper to showcase a document}
\author{Anna Author\affil{1}\and Bert Betatester\affil{2}\and Clara Creative\affil{2}}
\affiliation{
  \group{Thought Envisioning Group}\\
  \organization{Hasso Plattner Institute for Digital Engineering}\\
  \email{{firstname.lastname}@hpi.de}
  \and
  \group{Thought Envisioning Group}\\
  \organization{Hasso Plattner Institute for Digital Engineering}\\
  \email{{firstname.lastname}@hpi.de}
}
\keywords{paper, showcase, lorem ipsum}
\maketitle

\begin{abstract}
  Hello, here is some text without a meaning. This text should show
  what a printed text will look like at this place. If you read this
  text, you will get no information. Really? Is there no
  information? Is there a difference between this text and some
  nonsense like “Huardest gefburn”? Kjift – not at all! A blind text
  like this gives you information about the selected font, how the
  letters are written and an impression of the look. This text
  should contain all letters of the alphabet and it should be
  written in of the original language. There is no need for special
  content, but the length of words should match the language.
\end{abstract}


\section{Introduction}
\label{sec:introduction}

Hello, here is some text without a meaning. This text should show
what a printed text will look like at this place. If you read this
text, you will get no information. Really? Is there no information?
Is there a difference between this text and some nonsense like
“Huardest gefburn”? Kjift – not at all! A blind text like this gives
you information about the selected font, how the letters are written
and an impression of the look. This text should contain all letters
of the alphabet and it should be written in of the original
language. There is no need for special content, but the length of
words should match the language.\todo{cite}

\subsection{Contributions}
\label{sec:contributions}

Hello, here is some text without a meaning. This text should show
what a printed text will look like at this place. If you read this
text, you will get no information. Really? Is there no information?
Is there a difference between this text and some nonsense like
“Huardest gefburn”? Kjift – not at all! A blind text like this gives
you information about the selected font, how the letters are written
and an impression of the look. This text should contain all letters
of the alphabet and it should be written in of the original
language. There is no need for special content, but the length of
words should match the language.

\section{Context}
\label{sec:context}

Hello, here is some text without a meaning. This text should show
what a printed text will look like at this place. If you read this
text, you will get no information. Really? Is there no information?
Is there a difference between this text and some nonsense like
“Huardest gefburn”? Kjift – not at all! A blind text like this gives
you information about the selected font, how the letters are written
and an impression of the look. This text should contain all letters
of the alphabet and it should be written in of the original
language. There is no need for special content, but the length of
words should match the language.

\todosec{Is that redundant?}

\subsection{Background}
\label{sec:background}

Hello, here is some text without a meaning. This text should show
what a printed text will look like at this place. If you read this
text, you will get no information. Really? Is there no information?
Is there a difference between this text and some nonsense like
“Huardest gefburn”? Kjift – not at all! A blind text like this gives
you information about the selected font, how the letters are written
and an impression of the look. This text should contain all letters
of the alphabet and it should be written in of the original
language. There is no need for special content, but the length of
words should match the language.

\section{Problem}
\label{sec:problem}

\secmissing{The Main Problem}
Hello, here is some text without a meaning. This text should show
what a printed text will look like at this place. If you read this
text, you will get no information. Really? Is there no information?
Is there a difference between this text and some nonsense like
“Huardest gefburn”? Kjift – not at all! A blind text like this gives
you information about the selected font, how the letters are written
and an impression of the look. This text should contain all letters
of the alphabet and it should be written in of the original
language. There is no need for special content, but the length of
words should match the language.~\cite{953350}

\subsection{Specific Problem}
\label{sec:specific-problem}

Hello, here is some text without a meaning. This text should show
what a printed text will look like at this place. If you read this
text, you will get no information. Really? Is there no information?
Is there a difference between this text and some nonsense like
“Huardest gefburn”? Kjift – not at all! A blind text like this gives
you information about the selected font, how the letters are written
and an impression of the look. This text should contain all letters
of the alphabet and it should be written in of the original
language. There is no need for special content, but the length of
words should match the language.

\section{Solution}
\label{sec:solution}

\todolist{anton}{
\item Put A onto B
\item Put B into C
\item Pull D from C
}%
Hello, here is some text without a meaning. This text should show
what a printed text will look like at this place. If you read this
text, you will get no information. Really? Is there no information?
Is there a difference between this text and some nonsense like
“Huardest gefburn”? Kjift – not at all! A blind text like this gives
you information about the selected font, how the letters are written
and an impression of the look. This text should contain all letters
of the alphabet and it should be written in of the original
language. There is no need for special content, but the length of
words should match the language.

\subsection{Specific Solution}
\label{sec:specific-solution}

Hello, here is some text without a meaning. This text should show
what a printed text will look like at this place. If you read this
text, you will get no information. Really? Is there no information?
Is there a difference between this text and some nonsense like
“Huardest gefburn”? Kjift – not at all! A blind text like this gives
you information about the selected font, how the letters are written
and an impression of the look. This text should contain all letters
of the alphabet and it should be written in of the original
language. There is no need for special content, but the length of
words should match the language.

\begin{table}
  \centering
  \caption{Different Animals, different values}
  \label{tab:different}
  \begin{tabular}{@{}llSs@{}}
    \hiderowcolors
    \toprule
    \multicolumn{2}{c}{Item}\\ \cmidrule(r){1-2}
    Animal    & Description & {Value}   & Unit\\
    \midrule
    Gnat      & per gram &    2.3456 & \dB                       \\
              & each     &    1.2e-3 & \metre\squared\per\second \\
    Gnu       & stuffed  &       e3  & \kilo\hertz               \\
    Emu       & stuffed  &   90.473  & \percent                  \\
    Armadillo & frozen   & 5642.5    & \mega\byte                \\
    \bottomrule
    \showrowcolors
  \end{tabular}
\end{table}

\section{Implementation}
\label{sec:implementation}

Hello, here is some text without a meaning. This text should show
what a printed text will look like at this place. If you read this
text, you will get no information. Really? Is there no information?
Is there a difference between this text and some nonsense like
“Huardest gefburn”? Kjift – not at all! A blind text like this gives
you information about the selected font, how the letters are written
and an impression of the look. This text should contain all letters
of the alphabet and it should be written in of the original
language. There is no need for special content, but the length of
words should match the language.

\begin{figure}
  \centering
  \setlength{\unitlength}{.01in}
  \begin{picture}(200,75)
    \put(0,25){\vector(1,0){200}}
    \put(25,0){\vector(0,1){75}}
    \put(75,22){\line(0,1){6}}
    \put(125,22){\line(0,1){6}}
    \put(22,50){\line(1,0){6}}
    \thicklines
    \put(25,25){\line(1,0){50}}
    \put(75,50){\line(1,0){50}}
    \put(125,25){\line(1,0){72}}
    \put(17,50){\makebox(0,0){$1$}}
    \put(75,13){\makebox(0,0)[b]{$\pi$}}
    \put(125,13){\makebox(0,0)[b]{$2\pi$}}
    \put(195,13){\makebox(0,0)[b]{$t$}}
    \put(175,60){\makebox(0,0){$g(t)$}}
  \end{picture}
  \caption{A test figure}
  \label{fig:test}
\end{figure}

Hello, here is some text without a meaning. This text should show
what a printed text will look like at this place. If you read this
text, you will get no information. Really? Is there no information?
Is there a difference between this text and some nonsense like
“Huardest gefburn”? Kjift – not at all! A blind text like this gives
you information about the selected font, how the letters are written
and an impression of the look. This text should contain all letters
of the alphabet and it should be written in of the original
language. There is no need for special content, but the length of
words should match the language.

\missingfigure{Make a overview sketch of the whole system}

\subsection{Minor Detail}
\label{sec:detail}

\begin{algorithm}
\caption{The Bellman-Kalaba algorithm}
\begin{algorithmic}[1]
\Procedure {BellmanKalaba}{$G$, $u$, $l$, $p$}
\ForAll {$v \in V(G)$}
\State $l(v) \leftarrow \infty$
\EndFor
\State $l(u) \leftarrow 0$
\Repeat
\For {$i \leftarrow 1, n$}
\State $min \leftarrow l(v_i)$
\For {$j \leftarrow 1, n$}
\If {$min > e(v_i, v_j) + l(v_j)$}
\State $min \leftarrow e(v_i, v_j)$\State $p(i) \leftarrow v_j$
\EndIf
$+l(v_j)$
\EndFor
\State $l’(i) \leftarrow min$
\EndFor
\State $changed \leftarrow l \not= l’$
\State $l \leftarrow l’$
\Until{$\neg changed$}
\EndProcedure
\Statex
\Procedure {FindPathBK}{$v$, $u$, $p$}
\If {$v = u$}
\State \textbf{Write} $v$
\Else
\State $w \leftarrow v$
\While {$w \not= u$}
\State \textbf{Write} $w$
\State $w \leftarrow p(w)$
\EndWhile
\EndIf
\EndProcedure
\end{algorithmic}
\end{algorithm}

Hello, here is some text without a meaning. This text should show
what a printed text will look like at this place. If you read this
text, you will get no information. Really? Is there no information?
Is there a difference between this text and some nonsense like
“Huardest gefburn”? Kjift – not at all! A blind text like this gives
you information about the selected font, how the letters are written
and an impression of the look. This text should contain all letters
of the alphabet and it should be written in of the original
language. There is no need for special content, but the length of
words should match the language.

\section{Evaluation}
\label{sec:evaluation}

Hello, here is some text without a meaning. This text should show
what a printed text will look like at this place. If you read this
text, you will get no information. Really? Is there no information?
Is there a difference between this text and some nonsense like
“Huardest gefburn”? Kjift – not at all! A blind text like this gives
you information about the selected font, how the letters are written
and an impression of the look. This text should contain all letters
of the alphabet and it should be written in of the original
language. There is no need for special content, but the length of
words should match the language.

\lstset{language=Java}
\begin{lstlisting}[label=lst:example,caption={An example code snippet},float,numbers=left]
class HelloWorldApp {
    public static void main(String[] args) {
        System.out.println("Hello World!"); // Display the string.
    }
}
\end{lstlisting}

We have nice things: some code in \autoref{lst:example} and some
information in \autoref{tbl:things}.

\[\nabla \cdot \mathbf{E} = \frac{\rho}{\varepsilon_0}\]
\[\nabla \cdot \mathbf{B} = 0\]
\[\nabla \times \mathbf{E} = -\frac {\partial \mathbf{B}}{\partial t}\]
\[\nabla \times \mathbf{B} = \mu_0 \mathbf{J} + \mu_0\varepsilon_0  \frac{\partial \mathbf{E}}{\partial t}\]

\subsection{Threats to Valitidy}
\label{sec:threats-valitidy}

Hello, here is some text without a meaning. This text should show
what a printed text will look like at this place. If you read this
text, you will get no information. Really? Is there no information?
Is there a difference between this text and some nonsense like
“Huardest gefburn”? Kjift – not at all! A blind text like this gives
you information about the selected font, how the letters are written
and an impression of the look. This text should contain all letters
of the alphabet and it should be written in of the original
language. There is no need for special content, but the length of
words should match the language.

\section{Related Work}
\label{sec:related-work}

Hello, here is some text without a meaning. This text should show
what a printed text will look like at this place. If you read this
text, you will get no information. Really? Is there no information?
Is there a difference between this text and some nonsense like
“Huardest gefburn”? Kjift – not at all! A blind text like this gives
you information about the selected font, how the letters are written
and an impression of the look. This text should contain all letters
of the alphabet and it should be written in of the original
language. There is no need for special content, but the length of
words should match the language.

Hello, here is some text without a meaning. This text should show
what a printed text will look like at this place. If you read this
text, you will get no information. Really? Is there no information?
Is there a difference between this text and some nonsense like
“Huardest gefburn”? Kjift – not at all! A blind text like this gives
you information about the selected font, how the letters are written
and an impression of the look. This text should contain all letters
of the alphabet and it should be written in of the original
language. There is no need for special content, but the length of
words should match the language.

\section{Conclusion}
\label{sec:conclusion}

Hello, here is some text without a meaning. This text should show
what a printed text will look like at this place. If you read this
text, you will get no information. Really? Is there no information?
Is there a difference between this text and some nonsense like
“Huardest gefburn”? Kjift – not at all! A blind text like this gives
you information about the selected font, how the letters are written
and an impression of the look. This text should contain all letters
of the alphabet and it should be written in of the original
language. There is no need for special content, but the length of
words should match the language.

\begin{table}
  \centering
  \begin{threeparttable}
    \caption{Differences between things projected and things achieved}
    \label{tbl:things}
    \begin{tabular}{>{}p{.4\linewidth}@{}c}
      \toprule
      Part           & done \\
      \midrule
      Title          & yes  \\
      Abstract       & no   \\
      Intro          & yes  \\
      \midrule
      \multicolumn{2}{c}{Rest is not entirely true} \\
      \midrule
      Context        & yes  \\
      Problem        & no\tnote{a} \\
      Solution       & yes  \\
      Implementation & yes  \\
      Evaluation     & no   \\
      Related Work   & no   \\
      Conclusion     & yes  \\
      \bottomrule
    \end{tabular}
    \begin{tablenotes}
    \item [a] Just a few things missing
    \end{tablenotes}
  \end{threeparttable}
\end{table}

We have nice things: some code in \autoref{lst:example} and some
information in \autoref{tbl:things}.

Hello, here is some text without a meaning. This text should show
what a printed text will look like at this place. If you read this
text, you will get no information. Really? Is there no information?
Is there a difference between this text and some nonsense like
“Huardest gefburn”? Kjift – not at all! A blind text like this gives
you information about the selected font, how the letters are written
and an impression of the look. This text should contain all letters
of the alphabet and it should be written in of the original
language. There is no need for special content, but the length of
words should match the language.

\printbibliography
\backmatter
\end{document}
%%% Local Variables:
%%% mode: latex
%%% End:
